\documentclass[a4paper]{article}
\usepackage{graphicx}
\usepackage{xcolor} % Load this before hyperref
\definecolor{softblue}{RGB}{70,130,180}
\usepackage[colorlinks=true, linkcolor=softblue, citecolor=softblue, filecolor=softblue, urlcolor=softblue]{hyperref}
\usepackage{enumitem}


\title{\textbf{Assignment 2:\\Computer Architecture}}
\author{
    Yazeed AlKhalaf\\
    \\
    \textbf{Course:} CIS 304 - Computer Architecture\\
    \textbf{Instructor:} Dr. Adeel Baig
}
\date{\textbf{Date:} 21 Apr, 2024}

\begin{document}

\maketitle

\newpage

\tableofcontents

\newpage

\section{Question 1:}

A cache consists of 128 lines. The main memory contains 8192 blocks of 256
words each. Design the address format if the cache is

\begin{enumerate}[label=(\alph*)]
    \item Direct Mapped
    \item Associative Mapped
    \item Set Associative with four-line per sets
\end{enumerate}

\subsection{Shared Information:}

\begin{itemize}
    \item $cacheLines$ = 128 cache line
    \item $blocks$ = 8192 block
    \item $words$ = 256 word
    \item One memory address bit length: $\log_2(blocks * words)$ = $\log_2(8192*256)$ = $\log_2(2097152)$ = 21 bits
\end{itemize}

\subsection{Part A: Direct Mapped}

For the direct mapped cache, we need to know three things:

\begin{itemize}
    \item Required cache lines bit length: $\log_2(cacheLines)$ = $\log_2(128)$ = 7 bits
    \item Required word offset bit length: $\log_2(words)$ = $\log_2(256)$ = 8 bits
    \item Required tag bit length is the remaining bits length: $21 - 7 - 8$ = 6 bits
\end{itemize}

The design is: \textbf{6 Tag Bits | 7 Cache Line Bits | 8 Word Offset Bits}

\subsection{Part B: Associative Mapped}

Associative mapped cache means the cache can store any block in any line. Therefore, we only need to know the word offset bit length:

\begin{itemize}
    \item Required word offset bit length: $\log_2(256)$ = 8 bits
    \item Required tag bit length is the remaining bits length: $21 - 8$ = 13 bits
\end{itemize}

The design is: \textbf{13 Tag Bits | 8 Word Offset Bits}

\subsection{Part C: Set Associative with four-line per sets}

Set associative means that the cache lines are split into sets, and each set has some lines, in our case 4 lines, and those are associative which means 4 blocks can exist at the same set.

And here since our lines are split into sets of four, we need to get the number of sets to be able to calculate the "cache set" bit length. To calculate the number of sets, we do the following: ($cacheLines \div 4$). The $4$ here is inferred from the "four-line per set" part.

Taking all of that in mind, we can get the following values:

\begin{itemize}
    \item Number of cache sets: $128 \div 4$ = 32 cache sets
    \item Required "cache sets" bit length: $\log_2(32)$ = 5 bits
    \item Required word offset bit length: $\log_2(256)$ = 8 bits
    \item Required tag bit length is the remaining bits length: $21 - 5 - 8$ = 8 bits
\end{itemize}

The design is: \textbf{8 Tag Bits | 5 Cache Set Bits | 8 Word Offset Bits}

\section{Question 2:}

TODO :D

\end{document}